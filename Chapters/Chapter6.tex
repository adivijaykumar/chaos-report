% Chapter Template

\chapter{A Bound on Chaos} % Main chapter title

\label{Chapter6} % Change X to a consecutive number; for referencing this chapter elsewhere, use \ref{ChapterX}

\lhead{Chapter 6. \emph{A Bound on Chaos}} % Change X to a consecutive number; this is for the header on each page - perhaps a shortened title

Recall how we diagnose chaos in classical systems. For a dynamical variable q, we say that the system is chaotic if :- $$\frac{\partial q}{\partial q_0} = \{q(t),p_0 \} \sim e^{\lambda t},$$ where $q_0$ and $p_0$ are the initial conditions.

But in quantum mechanics, our dynamical variable is the wavefunction $\psi(x)4$, which respects unitary evolution. Due to this the inner product of the wavefunction will always have the same norm.

On thinking a bit more, one can easily see what's wrong with this analysis. We should ideally compare our wavefunction with not with the individual dynamical variables, but the phase space distribution $\rho(q,p)$.

Let us work for the moment with a spin system with spin centers $S_1, \ldots S_N$. At $t=0$, we have $[S_i,S_j]=0$. The system can be time-evolved using a Hamiltonian $H$, which we consider to be $k$-local and built out of these spin operators.


In quantum mechanics, we can use the following object for our investigations of chaos $$[W(t),V(0)]^2,$$ where 

